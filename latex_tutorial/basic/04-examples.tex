% arara: pdflatex
% arara: bibtex
% arara: pdflatex
% arara: pdflatex
\documentclass[a4paper,10pt]{article}

\usepackage{fullpage}
\usepackage{comment}

% font packages
\usepackage{palatino}
\usepackage{mathpazo}
%\usepackage{sansmathfonts}
%\usepackage[scaled]{helvet}
%\renewcommand*\familydefault{\sfdefault}

% packages for tables and lists
\usepackage{booktabs}
\usepackage{enumitem}

% mathematics packages
\usepackage{amsmath,amssymb,amstext,amsthm}

% package to include image
\usepackage{graphicx}
\graphicspath{ {./images/} }

% packge for bibliography
\usepackage[numbers]{natbib}
%\bibliographystyle{apalike}{}


\begin{document}

We can refer to the Sections \ref{s:intro} and \ref{s:duis} and also Subsection \ref{s:pretium}.
If we add extra sections or subsections in between, then the reference will automatically
update the next time we compile the source code.

\section{Introduction}
\label{s:intro}

Lorem ipsum dolor sit amet, consectetur adipiscing elit. Sed aliquam risus non augue hendrerit, a dapibus lectus
scelerisque. Nunc mollis id elit sed semper. Vestibulum et lectus risus. Vivamus luctus ante diam, a dapibus nisl
cursus eu. Sed at dignissim tortor, ac elementum eros. Cras sit amet tortor sed tortor ultrices tristique.
Mauris molestie ut libero eu euismod. Pellentesque ac faucibus lorem. Praesent sem urna, tincidunt vel nisl nec,
dignissim pulvinar lacus. Sed molestie ex sit amet sapien tincidunt lacinia. Integer varius laoreet erat, sit
amet maximus ante placerat ac.

Nunc efficitur tortor nec metus vestibulum posuere. Suspendisse nec pellentesque augue, ac iaculis odio.
In aliquet leo lorem, ut tristique ante ultrices nec. Integer sed lacinia purus. Vestibulum quis faucibus risus.
Ut faucibus vitae mi sit amet dapibus. In hac habitasse platea dictumst.

\section{Duis quam orci}
\label{s:duis}

Duis quam orci, lobortis sollicitudin lacinia in, lobortis a enim. Maecenas interdum auctor pulvinar.
Ut at mi id orci ultricies porta ultricies eu nulla. Sed tempor leo vel fringilla euismod. Donec gravida velit
eget tortor pretium malesuada. Pellentesque ut laoreet orci. Quisque feugiat, sapien sed cursus malesuada,
ante est tristique tellus, ut aliquam mi ligula vel nunc. Aliquam sagittis purus est, ut pretium justo cursus vitae.
Class aptent taciti sociosqu ad litora torquent per conubia nostra, per inceptos himenaeos.

Quisque sed mollis nisl. Vestibulum ex ligula, finibus non libero et, iaculis convallis nisi.
Integer vel erat sed erat ultrices ultrices sit amet sed leo. Morbi massa ex, maximus id lectus vitae,
hendrerit aliquam ante. Fusce feugiat eros sapien, sed convallis sapien egestas nec. Etiam aliquet at
nisi vitae feugiat. Proin convallis tempus nunc ac dapibus. Nulla consequat eget odio id bibendum.

\subsection{Pretium malesuada}
\label{s:pretium}

Duis quam orci, lobortis sollicitudin lacinia in, lobortis a enim. Maecenas interdum auctor pulvinar.
Ut at mi id orci ultricies porta ultricies eu nulla. Sed tempor leo vel fringilla euismod. Donec gravida velit
eget tortor pretium malesuada. Pellentesque ut laoreet orci. Quisque feugiat, sapien sed cursus malesuada,
ante est tristique tellus, ut aliquam mi ligula vel nunc. Aliquam sagittis purus est, ut pretium justo cursus vitae.
Class aptent taciti sociosqu ad litora torquent per conubia nostra, per inceptos himenaeos.
Maecenas quis leo a metus pharetra tincidunt. Etiam sed lectus vel mi semper vehicula a sed lorem.
Maecenas elementum sit amet enim et scelerisque.
Pellentesque habitant morbi tristique senectus et netus et malesuada fames ac turpis egestas.
Donec viverra, erat finibus interdum sollicitudin, massa enim venenatis ex, ac commodo dolor ipsum at urna.
Proin hendrerit sem ipsum, vel vestibulum risus fermentum id.

\pagebreak


\section*{List environments}

\subsubsection*{Standard itemize:}

\begin{itemize}
  \item at dignissim tortor
  \item ac elementum eros sit amet tortor sed tortor ultrices tristique
  \item mauris molestie ut libero eu euismod
  \item dignissim pulvinar lacus sed molestie ex sit amet sapien tincidunt
\end{itemize}

\subsubsection*{Modified bullet:}

\begin{itemize}
  \item[-] at dignissim tortor
  \item[*] ac elementum eros sit amet tortor sed tortor ultrices tristique
  \item[\checkmark] mauris molestie ut libero eu euismod
  \item[$\times$] dignissim pulvinar lacus sed molestie ex sit amet sapien tincidunt
\end{itemize}

\subsubsection*{Standard enumerate:}

\begin{enumerate}
  \item mauris molestie ut libero eu euismod
  \item ac elementum eros sit amet tortor sed tortor ultrices tristique
  \item at dignissim tortor
  \item dignissim pulvinar lacus sed molestie ex sit amet sapien tincidunt
\end{enumerate}

\subsubsection*{Roman numerals (using enumitem package):}

\begin{enumerate}[label=(\roman*)]
  \item mauris molestie ut libero eu euismod
  \item ac elementum eros sit amet tortor sed tortor ultrices tristique
  \item at dignissim tortor
  \item dignissim pulvinar lacus sed molestie ex sit amet sapien tincidunt
\end{enumerate}

\subsubsection*{Alphabetical (using enumitem package):}

\begin{enumerate}[label=\alph*)]
  \item mauris molestie ut libero eu euismod
  \item ac elementum eros sit amet tortor sed tortor ultrices tristique
  \item at dignissim tortor
  \item dignissim pulvinar lacus sed molestie ex sit amet sapien tincidunt
\end{enumerate}

\subsubsection*{Basic description:}

\begin{description}
  \item[mauris] molestie ut libero eu euismod
  \item[ac] elementum eros sit amet tortor sed tortor ultrices tristique
  \item[at] dignissim tortor
  \item[dignissim] pulvinar lacus sed molestie ex sit amet sapien tincidunt
\end{description}


\pagebreak


\section*{Math mode}
\subsubsection*{Inline mathematics}

\begin{quote}
where $Q(p)$ is the number of isomorphism classes ${\mathcal E}_p$ of elliptic curves over ${\mathbb F}_p$
such that for any curve $E \in {\mathcal E}_p$,
$|E({\mathbb F}_p)|$ and  $|\overline{E}({\mathbb F}_p)|$
are primes.
\end{quote}

\subsubsection*{Using equation environment}

\begin{equation}
  \label{math:a}
  \binom{u}{v} = \binom{u_{\ell-1},\dots,u_0}{v_{\ell-1},\dots,v_0}
\end{equation}

\begin{equation}
  \label{math:b}
  \begin{aligned}
    \left| \bigcup_{i=1}^m A_i \right| = \sum_{i=1}^{m} \left| A_i \right|
      &- \sum_{1\le i < j \le m} \left| A_i \cap A_j\right|
      + \sum_{1 \le i < j < k \le m} \left| A_i \cap A_j \cap A_k \right| \\
      &- \cdots 		\\
      &+ (-1)^{m-1}\left| A_1 \cap \cdots \cap A_m \right|
  \end{aligned}
\end{equation}


\subsubsection*{Using displaymath environment}
\noindent{}First equation:
\[
  \label{math:c}
  \binom{u}{v} = \binom{u_{\ell-1},\dots,u_0}{v_{\ell-1},\dots,v_0}
\]
Second equation:
\[
  \label{math:d}
  \begin{aligned}
    \left| \bigcup_{i=1}^m A_i \right| = \sum_{i=1}^{m} \left| A_i \right|
      &- \sum_{1\le i < j \le m} \left| A_i \cap A_j\right|
      + \sum_{1 \le i < j < k \le m} \left| A_i \cap A_j \cap A_k \right| \\
      &- \cdots 		\\
      &+ (-1)^{m-1}\left| A_1 \cap \cdots \cap A_m \right|
  \end{aligned}
\]
Third equation:
\[
  \label{math:e}
  \begin{aligned}
    x^2 - 2x + 1 &= (x-1)(x-1)\\
                 &= (x-1)^2
  \end{aligned}
\]

It is possible to refer to the math blocks, namely (\ref{math:a}) and (\ref{math:b}) above.
It is pointless to reference the displaymath ones because the counters are not incremented
(they are all numbered the same: (\ref{math:c}), (\ref{math:d}) and (\ref{math:e}) )
and they are not displayed on the equation anyway.

\subsubsection*{Most commonly used math symbols}:

Square roots, superscript, and subscript
\[
    \sqrt[3]{b_i + b_{i+1}^2}
\]

Sums and product:
\[
  \sum_{i=1}^k \left(\frac{i+1}{2}\right)^2
  \qquad\text{and}\qquad
  \prod_{\substack{i=1,\\ \text{$i$ is even}}}^k \left\{\frac{2^i}{3}\right\}
\]


\pagebreak
\renewcommand{\arraystretch}{1.4}
\section*{Tabular environment}
\ \\
\ \\

\begin{center}
\begin{tabular}{l | r r c | r r c }
\hline
\hline
  Name  & $x$ & $y$ & trend & $x$ & $y$ & trend \\
\hline
  Amanda  & 11.2 & 13.9 & increasing & 12.1 & 12.1 & none \\
  Daniel  & 14.5 & 11.4 & decreasing & 19.4 & 13.6 & decreasing \\
  Homer   &  0.0 &  0.0 & none       &  0.0 &  0.0 & none \\
\hline
\hline
\end{tabular}
\end{center}

\ \\
\ \\

\begin{center}
\begin{tabular}{l | r r c | r r c } \hline
  \multicolumn{1}{c}{}& \multicolumn{3}{c}{2015} & \multicolumn{3}{c}{2016} \\
  \multicolumn{1}{c}{Name}
  & \multicolumn{1}{c}{$x$} & \multicolumn{1}{c}{$y$} & \multicolumn{1}{c}{trend}
  & \multicolumn{1}{c}{$x$} & \multicolumn{1}{c}{$y$} & \multicolumn{1}{c}{trend} \\
  \hline
  Amanda  & 11.2 & 13.9 & increasing & 12.1 & 12.1 & none \\
  Daniel  & 14.5 & 11.4 & decreasing & 19.4 & 13.6 & decreasing \\
  Homer   &  0.0 &  0.0 & none       &  0.0 &  0.0 & none \\
\hline
\end{tabular}
\end{center}


\ \\
\ \\

\begin{center}
\begin{tabular}{l | r r c | r r c } \toprule[0.2pt]
  \multicolumn{1}{c}{}& \multicolumn{3}{c}{2015} & \multicolumn{3}{c}{2016} \\
  \multicolumn{1}{c}{Name}
  & \multicolumn{1}{c}{$x$} & \multicolumn{1}{c}{$y$} & \multicolumn{1}{c}{trend}
  & \multicolumn{1}{c}{$x$} & \multicolumn{1}{c}{$y$} & \multicolumn{1}{c}{trend} \\
  \midrule[2pt]
  Amanda  & 11.2 & 13.9 & increasing & 12.1 & 12.1 & none \\
  Daniel  & 14.5 & 11.4 & decreasing & 19.4 & 13.6 & decreasing \\
  Homer   &  0.0 &  0.0 & none       &  0.0 &  0.0 & none \\
\bottomrule[0.2pt]
\end{tabular}
\end{center}


\ \\
\ \\


\begin{center}
\begin{tabular}{l  r r c  r r c }
  & \multicolumn{3}{c}{2015} & \multicolumn{3}{c}{2016} \\
  \multicolumn{1}{c}{Name}
  & \multicolumn{1}{c}{$x$} & \multicolumn{1}{c}{$y$} & trend
  & \multicolumn{1}{c}{$x$} & \multicolumn{1}{c}{$y$} & trend \\
  \cmidrule[1.2pt](r){1-1} \cmidrule[1.2pt](lr){2-4} \cmidrule[1.2pt](l){5-7}
  Amanda  & 11.2 & 13.9 & increasing & 12.1 & 12.1 & none \\
  Daniel  & 14.5 & 11.4 & decreasing & 19.4 & 13.6 & decreasing \\
  Homer   &  0.0 &  0.0 & none       &  0.0 &  0.0 & none \\
\bottomrule[0.2pt]
\end{tabular}
\end{center}

\ \\
\ \\


\begin{center}
\begin{tabular}{l l}
  $A = Z_1^2$			&	$Z_3 = A\cdot C$ \\
  $B = b\cdot A^2$	&	$X_3 = C^2 + B$\\
  $C = X_1^2$			&  $Y_3 = (Y_1^2 + a\cdot Z_3 + B)\cdot X_3 + Z_3\cdot B$.
\end{tabular}
\end{center}


\pagebreak
\section*{Figures and tables}

\begin{figure}
\label{figure:here}
\begin{center}
\begin{tabular}{l | r r c | r r c }
\hline
\hline
  Name  & $x$ & $y$ & trend & $x$ & $y$ & trend \\
\hline
  Amanda  & 11.2 & 13.9 & increasing & 12.1 & 12.1 & none \\
  Daniel  & 14.5 & 11.4 & decreasing & 19.4 & 13.6 & decreasing \\
  Homer   &  0.0 &  0.0 & none       &  0.0 &  0.0 & none \\
\hline
\hline
\end{tabular}
\caption{Figure placed `here'}
\end{center}
\end{figure}

\begin{figure}[b!]
\label{figure:bottom}
\begin{center}
\begin{tabular}{l | r r c | r r c } \hline
  \multicolumn{1}{c}{}& \multicolumn{3}{c}{2015} & \multicolumn{3}{c}{2016} \\
  \multicolumn{1}{c}{Name}
  & \multicolumn{1}{c}{$x$} & \multicolumn{1}{c}{$y$} & \multicolumn{1}{c}{trend}
  & \multicolumn{1}{c}{$x$} & \multicolumn{1}{c}{$y$} & \multicolumn{1}{c}{trend} \\
  \hline
  Amanda  & 11.2 & 13.9 & increasing & 12.1 & 12.1 & none \\
  Daniel  & 14.5 & 11.4 & decreasing & 19.4 & 13.6 & decreasing \\
  Homer   &  0.0 &  0.0 & none       &  0.0 &  0.0 & none \\
\hline
\end{tabular}
\caption{Figure placed at bottom}
\end{center}
\end{figure}


\begin{table}[b!]
\label{table:bottom}
\begin{center}
\begin{tabular}{l | r r c | r r c } \toprule[0.2pt]
  \multicolumn{1}{c}{}& \multicolumn{3}{c}{2015} & \multicolumn{3}{c}{2016} \\
  \multicolumn{1}{c}{Name}
  & \multicolumn{1}{c}{$x$} & \multicolumn{1}{c}{$y$} & \multicolumn{1}{c}{trend}
  & \multicolumn{1}{c}{$x$} & \multicolumn{1}{c}{$y$} & \multicolumn{1}{c}{trend} \\
  \midrule[2pt]
  Amanda  & 11.2 & 13.9 & increasing & 12.1 & 12.1 & none \\
  Daniel  & 14.5 & 11.4 & decreasing & 19.4 & 13.6 & decreasing \\
  Homer   &  0.0 &  0.0 & none       &  0.0 &  0.0 & none \\
\bottomrule[0.2pt]
\end{tabular}
\caption{Table placed at bottom}
\end{center}
\end{table}



\begin{table}[t]
\label{table:top}
\begin{center}
\begin{tabular}{l  r r c  r r c }
  & \multicolumn{3}{c}{2015} & \multicolumn{3}{c}{2016} \\
  \multicolumn{1}{c}{Name}
  & \multicolumn{1}{c}{$x$} & \multicolumn{1}{c}{$y$} & trend
  & \multicolumn{1}{c}{$x$} & \multicolumn{1}{c}{$y$} & trend \\
  \cmidrule[1.2pt](r){1-1} \cmidrule[1.2pt](lr){2-4} \cmidrule[1.2pt](l){5-7}
  Amanda  & 11.2 & 13.9 & increasing & 12.1 & 12.1 & none \\
  Daniel  & 14.5 & 11.4 & decreasing & 19.4 & 13.6 & decreasing \\
  Homer   &  0.0 &  0.0 & none       &  0.0 &  0.0 & none \\
\bottomrule[0.2pt]
\end{tabular}
\end{center}
\caption{Table placed at top}
\end{table}

Duis quam orci, lobortis sollicitudin lacinia in, lobortis a enim. Maecenas interdum auctor pulvinar.
Ut at mi id orci ultricies porta ultricies eu nulla. Sed tempor leo vel fringilla euismod. Donec gravida velit
eget tortor pretium malesuada. Pellentesque ut laoreet orci. Quisque feugiat, sapien sed cursus malesuada,
ante est tristique tellus, ut aliquam mi ligula vel nunc. Aliquam sagittis purus est, ut pretium justo cursus vitae.
Class aptent taciti sociosqu ad litora torquent per conubia nostra, per inceptos himenaeos.
Maecenas quis leo a metus pharetra tincidunt. Etiam sed lectus vel mi semper vehicula a sed lorem.
Maecenas elementum sit amet enim et scelerisque.
Pellentesque habitant morbi tristique senectus et netus et malesuada fames ac turpis egestas.
Donec viverra, erat finibus interdum sollicitudin, massa enim venenatis ex, ac commodo dolor ipsum at urna.
Proin hendrerit sem ipsum, vel vestibulum risus fermentum id.



\begin{figure}[t]
\label{figure:top}
\begin{center}
\begin{tabular}{l l}
  $A = Z_1^2$			&	$Z_3 = A\cdot C$ \\
  $B = b\cdot A^2$	&	$X_3 = C^2 + B$\\
  $C = X_1^2$			&  $Y_3 = (Y_1^2 + a\cdot Z_3 + B)\cdot X_3 + Z_3\cdot B$.
\end{tabular}
\end{center}
\caption{Figure placed at top}
\end{figure}

\begin{figure}[t!]
\begin{center}
\includegraphics[width=10cm]{logo}
\caption{Example of image file in \TeX}
\end{center}
\end{figure}

Vestibulum ut pulvinar risus. Aliquam erat volutpat. Suspendisse iaculis ultricies neque nec hendrerit.
Donec vitae elit at urna laoreet facilisis. Vivamus commodo faucibus odio, vitae hendrerit lorem cursus eget.
Phasellus rhoncus aliquet lacus in bibendum. Morbi eleifend efficitur erat, non semper eros faucibus vel.
Nam eget cursus risus. Vivamus ut rutrum urna. Integer ac convallis felis.
Pellentesque lobortis maximus neque sit amet mattis.


Quisque \hspace{5cm} sed mollis nisl. Vestibulum ex ligula, finibus non libero et, iaculis convallis nisi.
Integer vel erat sed erat ultrices ultrices sit amet sed leo. Morbi massa ex, maximus id lectus vitae,
hendrerit aliquam ante. Fusce feugiat eros sapien, sed convallis sapien egestas nec. Etiam aliquet at
nisi vitae feugiat. Proin convallis tempus nunc ac dapibus. Nulla consequat eget odio id bibendum.



\pagebreak
\section*{Bibliography}

Examples of natbib reference styles:
\begin{itemize}
  \item using \textbackslash citet\{AdikariImbert\}: \citet{AdikariImbert}
  \item using \textbackslash citet*\{AdikariImbert\}: \citet*{AdikariImbert}
  \item using \textbackslash citep\{AdikariImbert\}: \citep{AdikariImbert}
  \item using \textbackslash citep\{AdikariImbert,Yao76\}: \citep{AdikariImbert,Yao76}
  \item using \textbackslash citep[see also][chapter 2]\{Yao76\}: \citep[see also][chapter 2]{Yao76}
\end{itemize}

\bibliographystyle{IEEEtranN}{}
\bibliography{references.bib}


% uncomment this for olde-style bibliography
\begin{comment}
The simplest approach for bibliography is to add a bibliography environment and manually enter
the references one by one. Document can then be cited like so \cite{KLMMSS}] and \cite{Mort}.

\begin{thebibliography}{9999}
\bibitem{KLMMSS}
 S. V.~Konyagin, F. Luca, B. Mans, L.  Mathieson, M.~Sha and I. E. Shparlinski,
 `Functional graphs of polynomials over finite fields',
{\it J. Combin. Theory, Ser. B\/}, {\bf 116} ( 2016), 87--122.
\bibitem{Mort}
P. Morton, `On certain algebraic curves related to polynomial maps',
{\it Compositio Math.\/}, {\bf 103}(1996), 319--350;
Corrigendum,  {\it Compositio Math.\/}, {\bf  147} (2011), 332--334.
\bibitem{Silv}
J. H. Silverman, {\it The arithmetic of dynamical systems},
Springer Verlag, 2007.
\end{thebibliography}
\end{comment}

\end{document}
